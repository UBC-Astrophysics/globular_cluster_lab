\documentclass{article}
\usepackage[pdftex]{graphicx}
\graphicspath{{../images/}}
\usepackage{boxedminipage}
\usepackage{ms}
\usepackage{tikz}
\usepackage[utf8]{inputenc}
\begin{document}
\begin{titlebox}{Translating MESA Results to Observables}
Ilaria Caiazzo, Jeremy Heyl \\
TAs: Xianfei Zhang, Sarafina Nance, Ilka Petermann
\end{titlebox}

\section{Introduction}

In the previous lab, you looked at the colour-magnitude diagram of your globular cluster with a set of two models.  One had solar composition, and one had  one-tenth of solar composition.  In this lab you will use the set of models that you and your colleagues calculated over the past couple hours to find the best model for the HST data for your cluster.  You will also use the solar composition models to study the Hyades and Pleiades open clusters with GAIA data.

\section{Analysing the Globular Cluster}

\section{Analysing the Open Clusters}

At the bottom of the notebook, you can plot the models at solar composition for the various masses to try to find the turn-off mass for the Hyades and Pleiades open clusters.   Here the data from GAIA gives the absolute magnitude of the stars, and these clusters are so close that reddening is not important, so you can plot the painted models directly on the data without any shifts.   Can you estimate the ages of these two clusters?


\end{document}
