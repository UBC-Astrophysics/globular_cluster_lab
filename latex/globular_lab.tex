\documentclass{article}
\usepackage[pdftex]{graphicx}
\graphicspath{{../images/}}
\usepackage{boxedminipage}
\usepackage{ms}
\usepackage{tikz}
\usepackage[utf8]{inputenc}
\begin{document}
\begin{titlebox}{Exploring Cluster Data}
Ilaria Caiazzo, Jeremy Heyl \\
TAs: Xianfei Zhang, Sarafina Nance, Ilka Petermann
\end{titlebox}

\section{Introduction}

White dwarfs are the final stage in the evolution of stars less than about eight solar masses.  Basically, all nuclear energy generation has ceased, so the stars shrink and cool.  

We have created several models of really young white dwarfs from the evolution of stars of one, two and four solar masses for you to explore the subsequent evolution.

We will look at how white dwarfs move through the Hertzsprung-Russell diagram and colour-magnitude diagrams as they age and also examine the luminosity function of white dwarfs, how many white dwarfs are there of a given brightness.

\section{The Inlist}

We will start with the test\_suite directory wd\_cool and change it for our initial models: 
\begin{verbatim}
    &star_job
      show_log_description_at_start = .false.

      mesa_dir = '../../..'

      save_model_number = 31
      save_model_filename = 'hires_surface.mod'

      load_saved_model = .true.
      saved_model_name = '../../../data/star_data/white_dwarf_models/1.025_from_
7.0_z2m2.mod'

      !pgstar_flag = .true.

/ ! end of star_job namelist
\end{verbatim}
We change the line with saved\_model\_name to use our first model from a one-solar-mass star (1Mz02wd.mod).  We will look at the other white dwarfs a bit later.

\section{The Evolution}

We will first focus on the white dwarf that results from a one-solar-mass star.  This is typical for the white dwarfs in globular clusters.

\begin{enumerate}
\item Let's first look at the track of luminosity against effective temperature.  What is happening to the star?
\item Let's look at luminosity against core temperature.  What is happening here?  What are the different regimes?
\item Let's look at luminosity against time.  What are the different regimes here?
\item 
Now run the evolution for the more massive white dwarfs and add their curves to the preceding diagrams.
\end{enumerate}

\section{Observations}

 

\end{document}
