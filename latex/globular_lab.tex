\documentclass{article}
\usepackage[pdftex]{graphicx}
\graphicspath{{../images/}}
\usepackage{boxedminipage}
\usepackage{ms}
\usepackage{tikz}
\usepackage[utf8]{inputenc}
\begin{document}
\begin{titlebox}{Exploring Cluster Data}
Ilaria Caiazzo, Jeremy Heyl \\
TAs: Xianfei Zhang, Sarafina Nance, Ilka Petermann
\end{titlebox}

\section{Introduction}

Globular clusters are groups of up to one million or more stars that were born at nearly the same time, with approximately the same composition and lie at a common distance from us.  This means that by today, the main parameter that will determine how the stars will appear is each of their masses.  To understand the population one must determine the shared age, distance and composition of the group of stars.   These properties of globular clusters (and open clusters as well) make them powerful laboratories to understand stellar evolution and new physics.

\section{The Inlist}

We will start with the \texttt{test\_suite} directory \texttt{1M\_pre\_ms\_to\_wd} and change it for our initial models.  Each student will choose an initial mass and initial metallicity to run and we will bring together all of the data in the next laboratory ``Translating MESA Results to Observables.''   Here is the list of models to choose from:

\begin{center}
\begin{tabular}{c|ccc}
\hline
$Z=0.02$ &
$Z=0.007$ &
$Z=0.002$ & 
$Z=0.0007$ 
\\  \hline 
3.0 & 1.1 & 1.1 & 1.1  \\
2.1 & 0.8 & 0.8 & 0.8  \\
1.5 & 0.6 & 0.6 & 0.6 \\
1.1 & 0.4 & 0.4 & 0.4  \\
0.8 & 0.3 & 0.3 & 0.3  \\
0.6 & 0.21 & 0.21 & 0.21 \\
0.4 & 0.15 & 0.15 & 0.15 \\
0.3 & 0.11 & 0.11 & 0.11 
\end{tabular}
\end{center}

Typically it takes MESA only a few minutes to reach the end of the central hydrogen burning but at least an hour to start helium burning and ascend the asymptotic giant branch, so if you start the models now, we will have a suite of models reaching up the giant branch and perhaps into the helium-burning regime.

\section{The Evolution}

We will first focus on the white dwarf that results from a one-solar-mass star.  This is typical for the white dwarfs in globular clusters.

\begin{enumerate}
\item Let's first look at the track of luminosity against effective temperature.  What is happening to the star?
\item Let's look at luminosity against core temperature.  What is happening here?  What are the different regimes?
\item Let's look at luminosity against time.  What are the different regimes here?
\item 
Now run the evolution for the more massive white dwarfs and add their curves to the preceding diagrams.
\end{enumerate}

\section{Observations}

 

\end{document}
