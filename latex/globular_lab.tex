\documentclass{article}
\usepackage[pdftex]{graphicx}
\graphicspath{{../images/}}
\usepackage{boxedminipage}
\usepackage{ms}
\usepackage{tikz}
\usepackage[utf8]{inputenc}
\begin{document}
\begin{titlebox}{Exploring Cluster Data}
Ilaria Caiazzo, Jeremy Heyl \\
TAs: Xianfei Zhang, Sarafina Nance, Ilka Petermann
\end{titlebox}

\section{Introduction}

Globular clusters are groups of up to one million or more stars that were born at nearly the same time, with approximately the same composition and lie at a common distance from us.  This means that by today, the main parameter that will determine how the stars will appear is their mass. To understand the population one must determine the shared age, distance and composition of the group of stars. These properties of globular clusters (and open clusters as well) make them powerful laboratories to understand stellar evolution and new physics.

\section{The Inlist}

We will start with the \texttt{test\_suite} directory \texttt{1M\_pre\_ms\_to\_wd} and change it for our initial models.  Each student will choose an initial mass and initial metallicity to run and we will bring together all of the data in the next laboratory ``Translating MESA Results to Observables.'' Here is the list of models to choose from:

\begin{center}
\begin{tabular}{c|ccc}
\hline
$Z=0.02$ &
$Z=0.007$ &
$Z=0.002$ & 
$Z=0.0007$ 
\\  \hline 
3.0 & 1.1 & 1.1 & 1.1  \\
2.1 & 0.8 & 0.8 & 0.8  \\
1.5 & 0.6 & 0.6 & 0.6 \\
1.1 & 0.4 & 0.4 & 0.4  \\
0.8 & 0.3 & 0.3 & 0.3  \\
0.6 & 0.21 & 0.21 & 0.21 \\
0.4 & 0.15 & 0.15 & 0.15 \\
0.3 & 0.11 & 0.11 & 0.11 
\end{tabular}
\end{center}

Typically it takes MESA only a few minutes to reach the end of the central hydrogen burning but at least an hour to start helium burning and ascend the asymptotic giant branch, so if you start the models now, we will have a suite of models reaching up the giant branch and perhaps into the helium-burning regime.

{\bf Task:}
\begin{enumerate}
 \setlength\itemsep{0em}
    \item 
Set \texttt{initial\_mass} and \texttt{initial\_z} equal to your assigned values in the table.  
\item Add  \texttt{envelope\_fraction\_left\_limit = 0.05} in the \texttt{star\_controls} section of the \texttt{inlist} file. 
\item Make and run MESA.
\end{enumerate}
We will use these models in the second part of this morning's lab.

\section{HST Data}

\textbf{Task:} We have a prepared a sample Jupyter notebook to analyse the data from the globular cluster 47 Tucanae and compare them with some precalculated models.   
\begin{enumerate}
 \setlength\itemsep{0em}
\item
In the directory in which you uncompressed this assignment, run \texttt{jupyter-notebook}.
\item
A web browser window will open, double-click on \texttt{Globular Cluster Lab.ipynb}.  This opens up the Jupyter notebook.
\item
Look through the notebook to see how to plot the data and the models. 
\end{enumerate}
We have included data for the globular cluster 47 Tucanane as an example along with the painted MESA history files for  $0.9~M_\odot$, $1.0~M_\odot$ and $1.4~M_\odot$ stars to compare with the data. We use the term ``painting'' to mean that we have included the calculations of the stellar atmospheres appropriate for the model to determine the stellar spectra and absolute magnitudes in a suite of filters.

The photometry data from the Hubble Space Telescope for many globular clusters (along with dwarf galaxies, Andromeda etc.) are available from the Hubble website.

\textbf{Task:}
You will download photometry for a cluster of your choice from the Hubble website.
\begin{enumerate}
 \setlength\itemsep{0em}
\item Go to https://archive.stsci.edu/prepds/acsggct/.  
\item Click on ``via browser.''
\item Choose a cluster to study.
\item Download the \texttt{.zpt} file from the directory.  
\end{enumerate}
This contains the magnitudes of the stars in the image in the HST filters F606W and F814W.  They are called V and I in the file from the HST website and F606W and F814W in the painted history files.


\end{document}
